\section{Implementation plan}
This section shows how the various components will be developed
and the timing constraints they will need. 

From Figure \ref{fig:component_diagram}, it can be clearly seen that the
components don't have specific dependencies, so the development process can be
done in parallel. This is especially true if we consider that all the components
use RPC for internal communication. This means that once one component has been
developed, it can be immediately unit tested, even the methods that require
communication with other components can be tested, this can be done by creating
some \emph{ad-hoc stubs} with the required interface.

Specifically, the order in which the various components are implemented is not
fixed, except for \emph{Router}, \emph{SuggestionsEngine} and
\emph{DBMS/FileServer (in case the MAKE option is preferred)}. The main reason
behind these choices is that, for example, the Router is responsible only of
taking care of the incoming requests and forward them to the correct component,
so it's role isn't critical in the integration of the components, plus it's
developing process is quite trivial. As for the storage systems (DBMS and File
Server), it is a common practice to test the components using volatile memory,
so a working database is not needed from the very beginning. Note that this does
not mean that the model of our world won't be designed before the implementing
phase. Indeed in this document an ER Diagram of the DB is provided. As for
SuggestionsEngine, its complexity compels us to start its implementation as soon
as possible. It's a machine learning based component so the sooner it is ready,
the better it is, because it can start its training and be truly ready for
deployment.

Finally despite the fact that the order, as previously stated, is not important,
some components should be implemented in pairs. It's the case of \emph{Violation
Collector} and \emph{ViolationManager}, \emph{MunicipalityAPIWrapper} and
\emph{AccidentDataCollector}, \emph{StatisticsEngine} and
\emph{StatisticsRepresentationManager}. This can ben necessary as these pairs of
components work closely and their integration should be as easy as possible.