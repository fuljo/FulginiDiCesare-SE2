\section{Functionalities and features}
Here is a table showing the various functionalities of the system and their
respective importance for the customer and difficulty of implementation. Note
that the \emph{Importance for the customer} isn't in any way related to the
importance for the system by a structural point of view. For example the Login
feature is very important for \emph{SafeStreets} since we truly need to register
our users, but from the point of view of the customer, it's something that won't
really be noticed. This table will be useful when the development will be
scheduled in details. The functionalities that are harder to implement will
require more resources and time than the trivial ones, furthermore we can note
that ML has a low importance for the customer while the implementation
difficulty is high, so a possible option is to launch the product without that
functionality and then release it later on.

\begin{table}[ht]
    \hyphenpenalty=100000 % Disable hyphenation
    \sffamily
    \rowcolors{1}{white}{lightgray}
    \begin{tabu} to \linewidth {X[2,l,m] X[1,c,m] X[1,c,m]}
    \toprule
    \rowfont{\bfseries}
    Feature & Importance for the customer & Difficulty of implementation \\
    \midrule
    Registration and Login & Low & Low \\ 
    Violation Report & High & Low \\ 
    Automatic metadata insertion & Medium & Low \\ 
    Automatic License Plate detection & Medium & High \\ 
    Visualize statistics & High & High \\ 
    Manage Violation Reports & High & Medium \\ 
    Collection of accidents data & Medium & Medium \\ 
    Visualize and manage suggestions & High & Low \\ 
    ML algorithm to generate suggestions & Low & High \\
    \bottomrule
    \end{tabu}
\end{table}

The development will follow a bottom-up approach, in order to improve components
reliability and progressively integrate the single components into the system.
Here is a list that describes which components will take care of each feature.
\begin{description}
    \item[Registration and Login] the main component for this feature is
        \emph{AuthenticationManager}, also the \emph{DBMS} will be needed in
        order to save the user's data.
    \item[Violation Report] for this feature \emph{ViolationManager} 
        and \emph{ViolationCollector} are needed, as well as \emph{DBMS}.
    \item[Automatic metadata insertion] this is a minor feature and it is taken
        care by the \emph{UserMobileApp}.
    \item[Automatic License Plate detection] this feature will be handled by
        \emph{LicensePlateExtractor} and will also involve the
        \emph{DMVAPIWrapper}.
    \item[Visualize Statistics] both users and authorities will have access to
        this feature, it needs \emph{StatisticsEngine} and
        \emph{StatisticsRepresentationManager}, the rendering of the data is
        performed by \emph{UserMobileApp}, or the \emph{AuthorityWebApp}. In any
        case, these components have a little work to do. In order to access
        stored data, also the \emph{DBMS} will be necessary.
    \item[Manage Violation Reports] in order to make this feature work, the
        component which is needed is \emph{ViolationManager}, and of course the
        \emph{DBMS}.
    \item[Collection of accidents data] for this feature the system will need
        the \emph{AccidentDataCollector} implemented. The data will be stored
        into the database, so also \emph{DBMS} is required.
    \item[Visualize and manage suggestions] this is the core functionality of
        \emph{SmartSuggestions} sub-system. The component in charge of this
        feature is \emph{SuggestionsManager}, which fetches data from the
        \emph{DBMS}.
    \item[Machine Learning algorithm to generate suggestions] this is probably
        the most difficult feature to implement. So we will need to take
        specific care, the component is \emph{SuggestionsEngine}, as well as the
        \emph{DBMS} to retrieve and store data.
\end{description}