\section{Document structure}
The present document is divided into chapters. Here we give a brief description
of each chapter and its intended audience.
\begin{description}
    \item[Chapter \ref{ch:intro}] introduces the document and gives a general
    idea on the contents that will be found in the following chapters. Contains
    also the revision history, references and the abbreviations and acronyms.
    \item[Chapter \ref{ch:arcdes}] is the core chapter of the document. In the
    first part gives a general idea on how the system will be, showing a
    high-level view of the various parts and the layers. Then it goes into
    details with the Component Diagram, Interface Diagram and the Database
    Design. The Runtime View then describes how the components work together
    during the main tasks. Finally the architectural choices
    are shown. This chapter is intended for \emph{back-end developers} and the
    \emph{sysadmin} as it shows how the back-end works and how the various
    application components interact.
    \item[Chapter \ref{ch:ui}] describes how the User Interface works and how it
    interacts with the user. The intended audience of this chapter are the
    \emph{front-end developers}.
    \item[Chapter \ref{ch:req}] provides a mapping between requirements and the
    components in charge to satisfy them. Each requirement is analyzed in
    details and also a brief description of how the component(s) will satisfy
    the requirement is given. Then there is a section that discusses the
    non-functional requirements and how the architectural choices fulfill them.
    This chapter is intended for \emph{developers}, \emph{testers} and also for
    \emph{designers}.
    \item[Chapter \ref{ch:test}] outlines a general schedule of implementation
    and testing. It starts with a general description of the various
    functionalities and how difficult they are to implement and integrate, then
    the implementation plan is discussed in details. Finally there is a section
    dedicated to the testing phase, beginning with code inspection and then
    ending with unit and integration test. The intended audience for this
    chapter are \emph{testers}, \emph{developers} and the \emph{Project Manager}
    who can use this chapter as a reference to check whether the project
    development is sticking to the plan.
\end{description}