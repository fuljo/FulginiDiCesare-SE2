\section{Runtime View}
In this section all the basic interactions will be described. All the use cases
from the RASD document are visualized in a deeper way. The Application Server is
exploded in the various components, just like in Figure
\ref{fig:component_diagram}, and also the interaction between the components is
described.

Regarding the methods exposed by the components, please refer to
section~\ref{sec:component_interfaces} where the interfaces of the main
components are described. Note however that also calls to internal methods of
the components are shown (always as self-arcs), so they are not included in the
interfaces.
Also note that the methods called by human actors on the clients are just
\emph{placeholders} intended to model the interaction.
The interaction with the DBMS is modeled with the SQL verbs \texttt{select},
\texttt{insert}, \texttt{update} and \texttt{delete}.

\begin{figure}[ht]
    \centering
    \includegraphics[width=\textwidth]{dd_sequence_diagram_uc_1_1}
    \caption{Registration process for a new User}
    \label{fig:dd_sequence_diagram_uc_1_1}
\end{figure}

\clearpage

\begin{figure}[ht]
    \centering
    \includegraphics[width=\textwidth]{dd_sequence_diagram_uc_1_2}
    \caption{Login process for a User}
    \label{fig:dd_sequence_diagram_uc_1_2}
\end{figure}

The login process for the operator is analogous to the one performed
by the user, shown in figure~\ref{fig:dd_sequence_diagram_uc_1_2}.
The only difference is that the operator interacts with the
\emph{AuthorityWebApp} instead of the \emph{UserMobileApp}.


\begin{figure}[ht]
    \centering
    \includegraphics[width=\textwidth]{dd_sequence_diagram_uc_1_3}
    \caption{Violation Report process for a User}
    \label{fig:dd_sequence_diagram_uc_1_3}
\end{figure}

\clearpage

\begin{figure}[ht]
    \centering
    \includegraphics[width=\textwidth]{dd_sequence_diagram_uc_1_4}
    \caption{Visualize Statistics process for a User}
    \label{fig:dd_sequence_diagram_uc_1_4}
\end{figure}

Note that Figure \ref{fig:dd_sequence_diagram_uc_1_4} shows the interaction
between Safe Streets and the User but the same action can be done by the
Authority Operator as well. In the latter case, there are no substantial
differences from the former case, so another dedicated Sequence Diagram is not
necessary.

\begin{figure}[ht]
    \centering
    \includegraphics[width=\textwidth]{dd_sequence_diagram_uc_2_1}
    \caption{Violation management process for the Authority}
    \label{fig:dd_sequence_diagram_uc_2_1}
\end{figure}

\begin{figure}[ht]
    \centering
    \includegraphics[width=\textwidth]{dd_sequence_diagram_uc_2_3}
    \caption{SmartSuggestions management for the Authority}
    \label{fig:dd_sequence_diagram_uc_2_3}
\end{figure}

\clearpage

\begin{figure}[ht]
    \centering
    \includegraphics[width=\textwidth]{dd_sequence_diagram_accident_collector}
    \caption{Periodic routines to fetch and process accident data}
    \label{fig:dd_sequence_diagram_accident_collector}
\end{figure}

Lastly, in Figure \ref{fig:dd_sequence_diagram_accident_collector}, we can see
how accident data is fetched from the Municipality API, and then how it is
processed from the SuggestionsEngine component. It is important to underline
that no notification will be pushed to the Authority Web App when new
Suggestions are available, they will be shown to the Operator once he selects
the \emph{SmartSuggestions} page.

