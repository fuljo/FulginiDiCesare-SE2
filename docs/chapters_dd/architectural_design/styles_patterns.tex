\section{Selected architectural styles and patterns}
\label{sec:styles_patterns}
\subsection{Four-tier client/server architecture}
As shown in section~\ref{sec:deployment_view}, we decided to use a client/server
architecture, where the components have been divided into four \emph{tiers}.
The communication is always initiated by the client, while the server only
waits for requests, processes them and finally responds. There are no cases
in which the server actively contacts the client (e.g.\ push notifications).
The client/server communication is fully based on HTTP.
Furthermore, in order to guarantee security, it is mandatory to use HTTP over
TLS (HTTPS) for such communication.

On the other hand the division in tiers enables to \emph{uncouple} the
components from one another, which facilitates parallel development, maintenance
and also hides the internal complexity of a tier (e.g.\ partitioning and
replication) to its users.

\subsection{RESTful communication and stateless components}
As already mentioned earlier, the communication between the client and the
application tier is based on the REST architectural style. Here both the mobile
and the web clients can perform operations (register, login, mark violations and
suggestions) and request data (violations, suggestions, statistics) in a
request-response fashion. In fact all the above mentioned functionalities are
designed in such a way that they can each be completed with a single request and
response. Furthermore each request also carries an access token (except
registration and login, of course) that is used to check the access rights of
the client who sent the request.
In this way the application components can be \emph{completely stateless} and
only rely on the data stored in the DB and file server.
This means that we can have multiple instances of the same component and the
\emph{Router}, implemented with a Kubernetes API Gateway, can forward an
incoming request to any of those instances.

Another advantage of using the RESTful architecture is that data can be
represented in a simple and platform-independent way using JSON, allowing for
a total uncoupling between the application server and the clients.
In fact, even if the citizen client is a native iOS or Android application,
while the authority client is a website, they can still share the same API,
as the conversion to the concrete data structures happens inside the client
itself.

\subsection{Elastic components and load balancers}
The choice of a RESTful architecture with stateless components means that a
request can be sent to any instance of a component, without affecting the
correctness of the response. These considerations of course hold also for
the static web server, which is stateless by definition.

We have exploited this advantage by running each component in a separate
\emph{Docker} container and all containers are managed by Kubernetes.
Kubernetes always keeps at least one instance of each component running, but
when the workload increases it automatically spawns new containers to replicate
stressed components and automatically assigns the requests to the various
instances in order to share the workload between them.
When the workload decreases again, the replicated containers are deleted
accordingly.

The goals of this decision are to achieve \emph{scalability} and also to make
an efficient use of the computational resources.

\subsection{Facade pattern}
The facade pattern consist in providing a unique and simplified interface to
interact with a complex system. This pattern has been used various times
in our design:
\begin{itemize}
    \item in the \emph{ApplicationServer} subsystem, which provides a unique
    REST API managed by the \emph{Router}. This hides the fact that the requests
    are actually fulfilled by other components, which may also be replicated
    other multiple nodes.
    \item in the \emph{data tier}, where both the Database and the File Server
    offer a single entrypoint to access data that is actually partitioned over
    multiple nodes.
\end{itemize}

\subsection{Adapter pattern}
\label{subsec:adapter_pattern}
This pattern consists in allowing an existing interface to be used as another
interface. We used this pattern on the \emph{DMVAPIWrapper} and
\emph{MunicipalityAPIWrapper}, both providing access to external APIs.
The external APIs are usually not consistent with the internal ones, in
terms of communication model (RPC in this case), data format (the data
structures may be different) and they may also change over time.
So the aim of the above mentioned components is to provide interfaces that are
consistent with the rest of the internal ones. If an external API changes for
some reason we only need to modify its wrapper, rather than all the components
that use it.

\subsection{Database sharding}
\label{subsec:db_sharding}
Database sharding is a database architecture pattern that consists in
\emph{horizontal partitioning}: separating the table rows into different
sub-tables (\emph{logical shards}) and assigning each of them to
a different node (\emph{physical shard})~\cite{digitalocean:db-sharding}.

This choice has been dictated by the fact that we expect that, at some point,
a single DB node wouldn't be able to store all the data, so we needed to apply
a \emph{horizontal scaling} technique that allows to efficiently partition the
data and share the workload between different nodes.

In order to actually balance the workload between the physical shards, a proper
partitioning criterion should be put in place.
We chose to create \emph{a logical shard for each authority}, meaning that
all the violations and suggestions regarding the municipalities under a specific
authority belong to the same shard. The rationales that led to this choice are:
\begin{itemize}
    \item a single authority has access to all and only the data about the
    municipalities under its jurisdiction
    \item the suggestions are computed on a per-municipality basis, so they
    only access data about a specific municipality
    \item the rate of requests for a single municipality (or a small subset)
    is relatively small and can be supported by a single node
    \item the rate of requests is (approximately) evenly spread between
    the municipalities
\end{itemize}
With this technique, the most common operations (retrieve suggestions and
violations, insert or update a single violation or suggestion) only require
access to a single partition, which makes them efficient.