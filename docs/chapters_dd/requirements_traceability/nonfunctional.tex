\section{Non-functional requirements}
The non-functional requirements, which comprise the \emph{performance
requirements}, \emph{design constraints} and \emph{software system attributes}
(sections 3.3-3.5 of the RASD), are particularly related to the \emph{deployment
strategy} (see section~\ref{sec:deployment_view}) and the \emph{architectural patterns}
(see section~\ref{sec:styles_patterns}) that have been applied.
Here we analyze in how the crucial aspects have been taken care of.

\paragraph{Scalability}
A lot of architectural decisions have been made to ensure the scalability of
the system:
\begin{itemize}[noitemsep]
    \item tiered architecture
    \item containerization of components with Docker and Kubernetes
    \item elastic replication and load balancing
    \item stateless components
    \item database sharding
\end{itemize}
These are also aimed at improving the \emph{reliability}, \emph{availability}
and reduce \emph{response times}.

\paragraph{Performance}
Aside from the response times for the clients, it has been required for the
suggestions to be generated periodically. As stated in
section~\ref{sec:component_view} the \emph{SuggestionsEngine} will take care of
scheduling the jobs to generate new suggestions, as new data becomes available.

\paragraph{Security}
The measures related to security include:
\begin{itemize}[noitemsep]
    \item firewalls to protect the internal components
    (section~\ref{sec:deployment_view})
    \item use of HTTPS for communication over the internet
    (section~\ref{sec:deployment_view})
    \item hashed passwords in the DB
    (sequence diagram~\ref{fig:dd_sequence_diagram_uc_1_1})
    \item session tokens to use the APIs (section~\ref{sec:component_view})
\end{itemize}

\paragraph{Mantainability}
The use of the RESTful architecture (section~\ref{sec:styles_patterns}) and the
loose coupling between the components are all aimed at providing a better
maintainability than a monolithic approach.
Furthermore the use of management tools such as Kubernetes provide a way of
managing and distributing images of the various components, which facilitates
the versioning and deployment processes.

\paragraph{Portability}
The \emph{mobile application} will be developed for both Android and iOS
platforms, keeping a consistent design language and user experience.
On the server side, the use of \emph{containerization} reduces the dependency
of the components from the hardware, since they run on a virtual environment.
This makes it easy to port the containers from one machine to another, provided
that they can run Docker and they have the same architecture (we can safely
assume to be dealing with x86/64 servers).