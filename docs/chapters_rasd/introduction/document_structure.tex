\section{Document structure}
The present document is divided into chapters. Here we give a brief description
of each chapter and its intended audience.
\begin{description}
    \item[Chapter \ref{ch:intro}] outlines the general purpose of the project,
    including the goals it has to reach. It illustrates the scope of the
    applications and provides some definitions that will be used throughout
    the document. It also contains the document revision history and the list of
    the specification documents that describe the project and the required
    analysis approach.
    It is intended for a \emph{general audience}, as the description is provided
    at a high and non-technical level.
    \item[Chapter \ref{ch:desc}] provides a description of the intended product.
    Here we illustrate the domain model by showing the main entities involved,
    their relations and life cycle.
    Then we illustrate the functionalities, the types of users and finally we
    set the boundaries of the domain.
    This chapter is addressed to the \emph{clients} for which the system is
    being built for, in order to check that the right functionalities and domain
    assumptions have been captured, but also to \emph{software engineers}
    and \emph{developers} that work on this project, so they can get a
    high-level description.
    \item[Chapter \ref{ch:reqs}] provides a detailed description of the
    requirements.
    Here we define the various interface requirements, the functional
    requirements (also illustrated with the help of use cases) and
    finally the non-functional requirements (performance, reliability, security
    and other constraints).
    The intended audience of this chapter are the \emph{designers},
    \emph{developers} and \emph{testers} of the project.
    When possible, we inserted some hints to achieve the desired requirements,
    which might become handy during the design phase.
    \item[Chapter \ref{ch:formal}] provides a formal analysis of the domain
    model using the \emph{Alloy} language.
    In particular it is aimed to show the consistency constraints of the data
    model.
    For this reason it is particularly addressed to the \emph{developers} of the
    data model and to the \emph{testers}, that will need to verify that the
    implemented product is consistent with the formal representation. 
\end{description}