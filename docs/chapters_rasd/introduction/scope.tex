\section{Scope}
SafeStreets is a service that can be offered and managed by authorities
to willing citizens who want to contribute to their city’s public order.
In this way the authorities can improve their control over traffic violations.

The service is supposed to be provided in Italy, so we assume that each
municipality has one reference authority (e.g.\ police department) that is
responsible for handling traffic violations in that area.
The authority may activate the service or not in its area.

% TODO: put a disegnino

First of all, the user must register to the service by providing his/her 
fiscal code, a username and a password of his choice.
By requiring the fiscal code to be unique for each user, we can avoid
duplicate accounts for the same citizen.

When the user detects a violation, he can open the software on his mobile 
device and take a picture. The system will elaborate the picture in order to 
auto-detect the licence plate number. Furthermore, the app will get from the 
device the date, time and GPS position. The auto-collected data will be shown 
to the user who can correct it before being sent to the authority.
The only data the user must manually insert is the infraction type, which 
will be chosen from a finite list of possible types.
This process will be designed to be as smooth as possible, to offer a clean 
user experience.

From the authority point of view, the violation reports are handled by
an operator, who is identified by an username (must be unique) and
a password.
The operator will review the violation and then decide whether to accept it.
He may discard the violation when it contains erroneous or incomplete data.
For example the photo may simply not show a violation, or it may happen that
the licence plate in the picture is partially unreadable and the number
inserted by the user is likely to be wrong.

After accepting the violation, the authority might schedule intervention
on the ground (e.g.\ remove the vehicle from the road).
Scheduling the intervention, however, is out of the scope of the software
to be.

The data collected is of course persistent as the S2B will offer various
statistics to citizens and authorities.
The authorities will be able to view statistics on the number and types of
violations.
The statistics can be filtered by violation type, location, date/time and
vehicle.
Citizens will also be able to view statistics, but in order to keep privacy
they won't be able to see the vehicles involved.

For the data about accidents, we assume that a municipality may provide it
or not. Therefore the function which suggest actions to prevent accidents
can only be offered in municipalities that provide such data.
The data is supposed to contain the location and accident type for each accident.
We also assume that there are a finite pre-defined set of accident types.