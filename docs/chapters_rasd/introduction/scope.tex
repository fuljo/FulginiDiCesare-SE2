\section{Scope}
The SafeStreets application is offered to willing citizens of a city who want 
to contribute to their city’s public order. Meanwhile the service is also 
designed for authorities that want to improve their control over traffic 
violations. 
The service can be seen as an intermediary between the two parts: citizens 
and authorities.

%TODO: put a disegnino

First of all, the user must register to the service by providing his/her 
fiscal code and a password of his choice. The S2B will be deployed in 
Italy, so we can safely assume that the fiscal code can identify in a 
unique way every user.

The user after opening the application will be allowed to submit a picture 
with all the related data such as date, time, position, generic notes.

When the user detects a violation, it can open the software on his mobile 
device and take a picture. The system will elaborate the picture in order to 
auto-detect the licence plate number. Furthermore, the app will get from the 
device the date, time and GPS position. The auto collected data will be shown 
to the user who can correct it before being sent to the authority.
The only data the user must manually insert is the infraction type, which 
will be chosen from a finite list of possible violations.
This process will be designed to be as smooth as possible, to offer a clean 
user experience.

Once the violation has been sent, an operator of the authority will review
it and then decide whether to accept it. The operator may discard the
violation when it contains erroneous or incomplete data. For example the
photo may simply not show a violation, or it may happen that the licence
plate in the picture is partially unreadable and the number inserted by
the user is likely to be wrong.

After accepting the violation, the operator will be able to assign it a level
of priority, so that the authority will be able to efficiently schedule the
intervention. Scheduling the intervention, by the way, is outside the scope
of the S2B.

The data collected is of course persistent as the S2B will offer various
statistics to citizens and authorities.
The authorities will be able to view statistics which aggregate violations
of the same type, same location, same time, or same vehicle.
On the other hand citizens will be able to view statistics regarding only 
the location, time and infraction type.