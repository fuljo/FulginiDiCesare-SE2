\begin{usecase}{1.3}
    \textbf{Name: }Report Violation \\
    \textbf{Actors: } User \\
    \textbf{Entry Conditions:}
    \begin{enumerate}
        \item The user has opened the application on his device.
        \item The user has successfully logged in.
        \item The device is connected to the internet.
    \end{enumerate}
    \textbf{Event flow:}
    \begin{enumerate}
        \item The user taps on the \emph{Report violation} button.
        \item The application accesses the device's camera.
        \item The user takes a picture of the violation.
        \item The system extracts the licence plate number from the picture.
        \item The application automatically detects the current date, time
        and location of the device and fills the corresponding fields.
        \item The user selects the type of violation from a finite list and
        can also correct other fields.
        \item The user taps on the \emph{Send report} button.
        \item The system receives the Violation report, checks data correctness
        and shows the user a confirmation message.
    \end{enumerate}
    \textbf{Exit conditions: } The user correctly sends the Violation report \\
    \textbf{Exceptions: }
    \begin{enumerate}
        \item \textbf{The system hasn't been able to automatically detect the
        licence plate number from the photo:} the number will be manually
        inserted by the user. This doesn't prevent the procedure to complete.
        \item \textbf{The user didn't fill all the fields:} he's prompted to
        fill all the fields before tapping \emph{Send report} again.
        \item \textbf{The inserted data are in an invalid format:} a warning is
        shown and the data must be filled again before sending the report.
    \end{enumerate}
    \tcblower
    \centering
    \includegraphics[width=.7\linewidth]{rasd_sequence_diagram_uc_1_3}
\end{usecase}