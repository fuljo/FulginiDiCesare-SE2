\section{Design Constraints}

\subsection{Standards compliance}
Here we set some standards for data representation, that match the ones of the
planned development country: Italy.
\begin{itemize}
    \item Fiscal codes are represented as capitalized strings.
    (e.g.\ \texttt{MRTMTT25D09F205Z})
    \item Licence plates are represented as capitalized strings without spaces
    (e.g.\ \texttt{AA123BB}).
    \item Latitude and longitude are represented as decimal degrees (DD), which
    is the most common representation format in geographical information systems
    \cite{wiki:decimal-degrees}; in this way they can be conveniently stored as
    floating point numbers.
\end{itemize}

Since the system deals with personal data, in the form of fiscal codes and
licence plate numbers, it must comply to the General Data Protection Regulation,
which applies to all individual citizens of the European Union (EU) and the
European Economic Area (EEA).
The treatment of data is done by police forces, so the system is also subject to
the Italian law DPR-15/2018 \cite{gu:dpr-15/2018} that regulates this matter.

\subsection{Hardware limitations}
As described in section \ref{sec:ass_deps_constr} the service is accessed
from a smartphone app on the user side and from the web browser from the
authority side. In both cases the Internet is used to communicate between
the end user devices and the core of the system.

The smartphones are therefore required to have an internet connection, a
camera (needed to take pictures of violations) and a GPS sensor (to locate
the violation).
