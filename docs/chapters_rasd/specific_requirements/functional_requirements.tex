\section{Functional requirements}
In this section we start by analyzing the \emph{scenarios} and defining
the \emph{use cases} (and related sequence diagrams) for each user class
(citizens and authorities).
Then we programmatically show the requirements that the S2B must satisfy in
order to reach the goals.

\subsection{User}
\subsubsection{Scenarios}

% TODO: Specify scenarios

\paragraph{Scenario 1}
This is the description of the scenario.

\subsubsection{Use cases}
In figure \vref{fig:uc_diagram_user} the various use cases in which the user
participates are shown.
After that we describe each use case in further detail and provide a sequence
diagram which shows the various steps.

\begin{figure}[h]
    \centering
    \includegraphics[width=\textwidth]{rasd_use_case_diagram_user}
    \caption{Use case diagram for the user}
    \label{fig:uc_diagram_user}
\end{figure}

\clearpage

\begin{usecase}{1.1}
\label{uc:1.1}
\textbf{Name:} Register\\
\textbf{Actors:} User\\
\textbf{Entry conditions:}
\begin{enumerate}
    \item The user has opened the application on his device
    (he's in the \emph{landing page}).
    \item The user hasn't already logged in.
    \item The device is connected to the internet.
\end{enumerate}
\textbf{Event flow:}
\begin{enumerate}
    \item The user taps on the \emph{Register} button.
    \item The user is presented with a form, in which he inserts his
    fiscal code, username and desired password.
    \item The user taps on the \emph{Register} button to request the
    registration.
    \item The system receives the registration and stores the information
    entered by the user.
    \item The user receives a confirmation in the app that the registration
    terminated successfully.
\end{enumerate}
\textbf{Exit conditions:} The user is registered to the system and is now able
to log in.\\
\textbf{Exceptions:}
\begin{enumerate}
    \item\textbf{The fiscal code or the username already belongs to another user:}
    a warning is shown and the registration process is aborted.
    \item\textbf{The user inserted an invalid fiscal code:} a warning is shown.
    \item\textbf{The user didn't fill all the fields:} he's prompted to
    fill all the fields before tapping \emph{Register} again.
\end{enumerate}
\tcblower
\centering
\includegraphics[width=.7\linewidth]{rasd_sequence_diagram_uc_1_1}
\end{usecase}
\clearpage
\begin{usecase}{1.2}
\label{uc:1.2}
\textbf{Name:} Login\\
\textbf{Actors:} User\\
\textbf{Entry conditions:}
\begin{enumerate}
    \item The user has opened the application on his device.
    \item The user hasn't already logged in.
    \item The device is connected to the internet.    
\end{enumerate}
\textbf{Event flow:}
\begin{enumerate}
    \item The user taps on the \emph{Login} button.
    \item The user is presented with a form in which he inserts his username and
    password
    \item The user taps on the \emph{Login} button to send the login request.
    \item The system receives the login request and checks the inserted
    data.
    \item The user is now logged in and lands on the \emph{Main Menu} page.
\end{enumerate}
\textbf{Exit conditions: } The user is logged in and he's able to access all the
application services. \\
\textbf{Exceptions:}
\begin{enumerate}
    \item \textbf{The combination of username and password does not exist:} the
    user is shown an error pop-up and he's invited to insert the data again.
    \item \textbf{The user didn't fill all the fields:} he's prompted to fill
    all the fields before tapping the \emph{Login} button again
\end{enumerate}
\tcblower
\centering
\includegraphics[width=.7\linewidth]{rasd_sequence_diagram_uc_1_2}
\end{usecase}
\clearpage
\begin{usecase}{1.3}
    \textbf{Name: }Report Violation \\
    \textbf{Actors: } User \\
    \textbf{Entry Conditions:}
    \begin{enumerate}
        \item The user has opened the application on his device.
        \item The user has successfully logged in.
        \item The device is connected to the internet.
    \end{enumerate}
    \textbf{Event flow:}
    \begin{enumerate}
        \item The user taps on the \emph{Report violation} button.
        \item The application accesses the device's camera.
        \item The user takes a picture of the violation.
        \item The system gathers all the data it can and automatically fills the
        corresponding fields.
        \item The user fills the remaining data fields.
        \item The user taps on the \emph{Send report} button.
        \item The system receives the Violation report, checks data and shows
        the user a confirmation message.
    \end{enumerate}
    \textbf{Exit conditions: } The user correctly sends the Violation report \\
    \textbf{Exceptions: }
    \begin{enumerate}
        \item \textbf{The user didn't fill all the fields:} he's prompted to
        fill all the fields before tapping \emph{Send report} again.
        \item \textbf{The inserted data are in an invalid format:} a warning is
        shown and the data must be filled again before sending the report.
    \end{enumerate}
    \tcblower
    \centering
    \includegraphics[width=.7\linewidth]{rasd_sequence_diagram_uc_1_3}
\end{usecase}
\clearpage
\begin{usecase}{1.4}
    \textbf{Name: }Report Violation \\
    \textbf{Actors: } User \\
    \textbf{Entry Conditions:}
    \begin{enumerate}
        \item The user has opened the application on his device.
        \item The user has succesfully logged in.
        \item The device is connected to the internet.
    \end{enumerate}
    \textbf{Event flow:}
    \begin{enumerate}
        \item The user taps on the \emph{View statistics} button.
        \item The user is shown some options that describe the various
        statistics available.
        \item The system receives the request and answers with the available
        statistics type.
        \item The user selects one of the available statistics type.
        \item The system receives the request and answers with the statistics
        data, which are shown to the user.
    \end{enumerate}
    \textbf{Exit conditions:} The user visualizes the statistics. \\
    \textbf{Exceptions:}
    \begin{enumerate}
        \item \textbf{There are no statistics to be shown:} a warning is shown
        to the user.
    \end{enumerate}
    \tcblower
    \centering
    \includegraphics[width=.7\linewidth]{rasd_sequence_diagram_uc_1_4}
\end{usecase}

\clearpage

\subsection{Authority}
\subsubsection{Scenarios}

\paragraph{Scenario 2.1}
Milan's Local PD cannot afford to send every morning several traffic cops to
prevent people from parking on the crosswalks, there are more urgent works to
do! But at the same time this type of infraction cannot be backed off. To fix
this problem, the Local PD can use \emph{Safe Streets}. A Local PD operator can
see from his screen all the reports regarding that specific type of violation
and then take action against them.

\paragraph{Scenario 2.2}
The city of Pesaro has activated \emph{SafeStreets} in its area two years ago
and it has also installed a system to keep track of traffic-related accidents,
in order to take advantage of the \emph{SmartSuggestions} feature.
Recently a lot of car crashes have been registered in the intersection between
via Venezia and via Milano, mainly because the cars from via Venezia didn't give
the other ones right of way.
In the same period, \emph{SafeStreets} has received numerous reports of cars
parked at the corners of that intersections.
\emph{SmartSuggestions} is able to cross the data and infer that the parked cars
might have obstructed the view of the drivers coming from via Venezia, who
failed to see the other cars approaching. So it emits a suggestion to build
\emph{bollards} to physically block illegal parking at the corners of the
intersection.
An employee of the local police logs into SafeStreets on his office computer and
check if there are any suggestions and the system shows the newly created one.
After brief considerations, the Police orders to build the suggested bollards,
so another operator comes back to SafeStreets and marks the action as
\emph{carried out}.
After a few weeks no more accidents have occurred at the intersection, so
\emph{SmartSuggestions} marks the suggested action as useful, and it will likely
propose it again in similar situations.

\subsubsection{Use cases}
In figure \vref{fig:uc_diagram_authority} are shown the various use cases in
which the authority participates. Then each use case is described in a detailed
way with also a sequence diagram that shows the interaction steps.

\begin{figure}[h]
    \centering
    \includegraphics[width=.65\textwidth]{rasd_use_case_diagram_authority}
    \caption{Use case diagram for the authority}
    \label{fig:uc_diagram_authority}
\end{figure}

\clearpage
\begin{usecase}{2.1}
    \textbf{Name:} Review Violations \\
    \textbf{Actors:} Authority \\
    \textbf{Entry Conditions:}
    \begin{enumerate}
        \item The authority operator has opened the application on the device.
        \item The authority operator has already logged in.
        \item The device is connected to the internet.
    \end{enumerate}
    \textbf{Event flow:}
    \begin{enumerate}
        \item The authority operator clicks on the \emph{Violations} button.
        \item The authority operator is shown all the Violation Reports send by
        the users.
        \item The authority operator selects the Violation Report he wants to
        review.
        \item The system shows the options available for the Report: Accept or
        Discard.
        \item The authority operator selects the action he wants to take.
        \item The system elaborates the action and sends the operator a
        confirmation message.
    \end{enumerate}
    \textbf{Exit conditions:} The authority operator selects an action to be
    taken on the specific Violation Report \\
    \textbf{Exceptions:}
    \begin{enumerate}
        \item \textbf{There are no violation report to review:} the authority
        operator is shown a message stating that all the reports has been
        reviewed.
    \end{enumerate}
    \tcblower
    \centering
    \includegraphics[width=.7\linewidth]{rasd_sequence_diagram_uc_2_1}
\end{usecase}{2.1}
\clearpage
\begin{usecase}{2.2}
    \textbf{Name: } View Statistics \\
    \textbf{Actors: } Authority \\
    \textbf{Entry Conditions: }
    \begin{enumerate}
        \item The authority operator has opened the application on the device.
        \item The authority operator has already logged in.
        \item The device is connected to the internet.
    \end{enumerate}
    \textbf{Event flow:}
    \begin{enumerate}
        \item The authority operator clicks on the \emph{Statistics} button.
        \item The authority operator is shown all the available statistics type.
        \item The authority operator selects the statistics he wants to see.
        \item The system receives the request and elaborates the data.
        \item The authority operator is shown a graph that visualizes the
        statistics he chose.
    \end{enumerate}
    \textbf{Exit conditions:} The authority operator visualizes the statistics.
    \\
    \textbf{Exceptions:}
    \begin{enumerate}
        \item \textbf{There are no statistics to be shown:} a warning is shown
        to the authority operator.
    \end{enumerate}
    \tcblower
    \centering
    \includegraphics[width=.7\linewidth]{rasd_sequence_diagram_uc_2_2}
\end{usecase}
\clearpage
\begin{usecase}{2.3}
    \textbf{Name:} Manage Suggestions \\
    \textbf{Actors:} Authority \\
    \textbf{Entry Conditions:}
    \begin{enumerate}
        \item The authority operator has opened the application on his device.
        \item The authority operator has already logged in.
        \item The device is connected to the internet.
    \end{enumerate}
    \textbf{Event flow:}
    \begin{enumerate}
        \item The authority operator clicks on the \emph{Suggestions} button.
        \item The authority operator is shown all the available suggestions,
        generated by \emph{Smart Suggestions}.
        \item The authority operator clicks on the suggestion he wants to see in
        detail.
        \item The authority operator is shown all the details.
        \item The authority operator can click con \emph{Mark as carried out} or
        simply exit.
    \end{enumerate}
    \textbf{Exit conditions:} The authority operator visualizes the suggestion.
    \\
    \textbf{Exceptions:}
    \begin{enumerate}
        \item \textbf{There are no suggestions to be shown:} the authority
        operator is shown a message stating that there are no suggestions
        available.
    \end{enumerate}
    \tcblower
    \centering
    \includegraphics[width=.7\linewidth]{rasd_sequence_diagram_uc_2_3}
\end{usecase}

\subsection{Requirements}
Here we programmatically introduce the requirements needed to satisfy our goals.
In order to make apparent that the requirements satisfy the goals in the context
of domain assumptions, for each goal we will list the requirements and domain
assumptions that are needed to satisfy it.

\begin{description}
    % Goal 1
    \RuleItem{G1}
    \begin{description}
        \RuleItem{R1}
        \RuleItem{R2}
        \RuleItem{R3}
        \RuleItem{R4}
        \RuleItem{R5}
        \RuleItem{R6}
        \RuleItem{R7}
        \RuleItem{R8}
        \RuleItem{D1}
        \RuleItem{D2}
        \RuleItem{D3}
        \RuleItem{D4}
        \RuleItem{D8}
    \end{description}
    % Goal 2
    \RuleItem{G2}
    \begin{description}
        \RuleItem{R3}
        \RuleItem{R8}
        \RuleItem{R9}
    \end{description}
    % Goal 3
    \RuleItem{G3}
    \begin{description}
        \RuleItem{R10}
        \RuleItem{R11}
        \RuleItem{R12}
    \end{description}
    % Goal 4
    \RuleItem{G4}
    \begin{description}
        \RuleItem{R2}
        \RuleItem{R10}
        \RuleItem{R13}
        \RuleItem{R14}
        \RuleItem{R15}
        \RuleItem{R16}
        \RuleItem{D3}
        \RuleItem{D5}
        \RuleItem{D6}
        \RuleItem{D7}
        \RuleItem{D8}
    \end{description}
    % Goal 5
    \RuleItem{G5}
    \begin{description}
        \RuleItem{R2}
        \RuleItem{R10}
        \RuleItem{R13}
        \RuleItem{R17}
        \RuleItem{R18}
        \RuleItem{R19}
        \RuleItem{D8}
    \end{description}
    % Goal 6
    \RuleItem{G6}
    \begin{description}
        \RuleItem{R10}
        \RuleItem{R17}
        \RuleItem{R18}
        \RuleItem{R20}
        \RuleItem{D8}
    \end{description}
    % Goal 7
    \RuleItem{G7}
    \begin{description}
        \RuleItem{R10}
        \RuleItem{R21}
        \RuleItem{R22}
        \RuleItem{R23}
        \RuleItem{R24}
        \RuleItem{R25}
        \RuleItem{D5}
        \RuleItem{D6}
        \RuleItem{D7}
        \RuleItem{D8}
    \end{description}
\end{description}
