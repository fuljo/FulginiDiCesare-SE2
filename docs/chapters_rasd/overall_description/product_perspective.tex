\section{Product perspective}
The basic idea upon which the system will be built is that the user is a volounteer, so it has to put as little effort as possible to send information about violations.

Right after the user enters the application, it can take a picture directly from there. Then the taken picture is scanned by an OCR algorithm that detects (or at least tries to) the biggest licence plate and obtains its number. The application proceeds to collect as much data as possible from the built-in functions of the host device: gets the time, date and retrieves location through GPS. The auto-collected data is then showed to the user who can correct or complete it.
Finally the user inserts the infraction type and the violation is ready to be sent to the authorities. At this point the violation is still linked to the user which makes it, but as soon as the violation is sent to the authorities, it becomes anonymous. This difference can be clearly seen in the Class Diagram below. We have two subclasses of the superclass \emph{Violation}: \emph{UserEndViolation} and \emph{AuthorityEndViolation}. The latter is anonymous and is enriched with two states: accepted/discarded and the level of priority.

% Insert class diagram