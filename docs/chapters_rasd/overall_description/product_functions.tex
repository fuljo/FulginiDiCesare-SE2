\section{Product functions}
In this section the major functions of the S2B will be summarized.

\minisec{Report a Violation}
This is the core function of the system. When a user spots a violation, it can
open the app, log-in using username and password and rapidly take a picture. The
system will now run an OCR algorithm to find the biggest licence plate on the
picture and extract the number. Other information are required to complete the
violation report: date, time and position. They can be collected through the
standard tools of the mobile host device. If one or more of these information
are wrong or incomplete, then the user can correct the wrong ones. Finally the
user must select the right Violation Type. When a Violation report has been
completed, the user can choose if he wants to send it or to discard it.

Violations are automatically routed to the correct authority based on the
municipality they occurred in.

An authority operator can see all the violations the authority is responsible
for and review them. In particular he can accept the violation, or in case it is
considered inappropriate or simply erroneous, discard it.

\minisec{View Statistics}
Both users and operators can ask the system for statistics about the number and
type of violations.
Authorities and users can see different statistics, for privacy reasons a common
user cannot see data about the specific vehicles.
Some examples of available statistics are:
\begin{itemize}[noitemsep]
    \item Number of violations of a certain type
    \item Number of violations divided by area
    \item Number of violations divided by time of the day
    \item Number of violations divided by vehicle (\emph{authority only})
\end{itemize}
Crossed data statistics are also available (e.g.\ most popular violation type
for each area).

\minisec{SmartSuggestions}
If the municipality allows SafeStreets to collect their data about accidents,
the S2B will cross this data with the violations occurred in that area.
The system will try to determine a \emph{cause-effect} relation between
violations and accidents and finally suggest a smart suggestion to prevent that
accident from occurring again.

New suggestions will be displayed to the operator as they become available.
Furthermore the operator can mark the suggestions as carried out.
Since then the system will monitor the number of accidents of that type in that
area to determine if the suggestion was effective or not, and possibly provide
better suggestions in the future.