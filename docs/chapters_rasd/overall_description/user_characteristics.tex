\section{User characteristics}
Here we describe the various users of the applications, distinguishing
them by their role and the functions they have access to.

\begin{description}
    \item[User] a citizen, uniquely identified by a fiscal code, who registers
    to SafeStreets by providing username and password.
    After logging in he will be able to report violations to authorities
    and view anonymized statistics about violations.
    In order for him to be able to use the service he will need a mobile
    device (i.e.\ smartphone) with a camera and GPS.

    \item[Authority] a recognized government organization that enforces
    traffic laws in a certain area.
    The authority can ask to be registered to SafeStreets by certifying its
    status and providing the list of municipalities under its jurisdiction.
    The authority will receive a set of username/password pairs which will
    be given to its operators to enable them to access the system.
    If the municipalities in which the authority operates also provide
    data about accidents, the authority can request to also activate
    the \emph{SmartSuggestions} function by providing access to this
    data.
    
    \item[Operator] an employee of the authority, to which the authority
    provides an username/password pair to access the system.
    The \emph{basic service} of SafeStreets enables the operator to manage
    violations reported in the \emph{competence area} of his organization;
    this includes: viewing, accepting and discarding violations,
    viewing detailed statistics about the violations (including the
    involved vehicles). 
    If the \emph{SmartSuggestions} feature is enabled for that authority
    he will also be able to view suggestions generated by the system and
    manage them.
    Both functionalities can be accessed from a web browser.

\end{description}