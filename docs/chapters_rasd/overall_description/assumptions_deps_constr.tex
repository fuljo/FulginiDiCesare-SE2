\section{Assumptions, dependencies and constraints}
\label{sec:ass_deps_constr}

Here we present the domain assumptions that must hold in order for the system
to work properly.
\begin{description}
    \RuleItem{D1}
    \RuleItem{D2}
    \RuleItem{D3}
    \RuleItem{D4}
    \RuleItem{D5}
    \RuleItem{D6}
    \RuleItem{D7}
    \RuleItem{D8}
\end{description}

\minisec{Hardware requirements}
In order to use the service, the user will need a smartphone with a camera,
a GPS sensor and internet connectivity. Both Wi-Fi and data connections are fine
as long as they provide a reasonable bandwidth to upload and download data.

The operators of the authorities can access their dedicated functions through
a web browser. In this way the service can be accessed both from desktop
computers at the authorities' headquarters and from the mobile devices of the
on-field agents. In both cases a stable internet connection is required.

\minisec{Data about accidents}
In case the \emph{SmartSuggestions} function is to be activated, the authority
must also provide access to the data about accidents in the municipality(ies)
under its jurisdiction.
The assumption is that the data can be retrieved in a structured or
semi-structured format (e.g.\ JSON, XML) through a REST API.
The API must allow to retrieve the data about the accidents of a single
municipality, indicating the location (e.g.\ latitude, longitude, address) and
type of each accident (within a finite set of pre-defined types).
Furthermore it must allow to retrieve the violations that occurred after a
certain date, in order for SafeStreets to perform incremental updates.

\minisec{Privacy constraints}
According to Article 4 of the General Data Protection Regulation
\cite{gdpr:article-4-definitions} both the fiscal code and the licence plate
number are considered \emph{personal data} as they can be directly or indirectly
associated to a person. Therefore our system must comply to the GDPR, and in
particular to the national law DPR 15/2018 \cite{gu:dpr-15/2018} that regulates
the treatment of personal data by police forces.